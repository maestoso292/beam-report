\documentclass[../report.tex]{subfiles}
\begin{document}
\section{Summary of Achievements}
We are proud to note that we have successfully managed to complete, run and test our application according to our stated user and functional requirements. According to our requirements, the app takes the log in information of the student or lecturer, opens the student mode or the lecturer mode of the BEAM app, then allows the user to take attendance automatically when the lecturer sends the attendance token, then the student device takes attendance and sends out an attendance token itself in a Waterfall system until no more devices with that timeslot are checked for in the area. Relevant notifications are sent to the lecturer and to the student:
\begin{enumerate}
\item Lecturer
\begin{itemize}
\item Opening Attendance
\item Advertising Tokens for MODULE\_CODE
\item Closing Attendance
\end{itemize}
\item Student
\begin{itemize}
\item Taking Attendance
\item Attendance Taken
\item Advertising Tokens for MODULE\_CODE
\end{itemize}
\end{enumerate}

For the display and navigation requirements, the students can see their relevant daily timetables when they swipe right on the main screen, if they swipe right again, they can see their weekly timetables and if they swipe right again they can see their attendance records successfully. There is also a log out button that allows users to log out. Only one user may use the phone to log in and register for a class, it does not allow the attendance token to take the attendance of multiple accounts, thereby limiting the amount of cheating currently done by students with attendance during class. 

The lecturer on the other hand can schedule background services for opening and closing attendancefor each of their relevant classes. Their timetables can be seen same as the students displays above. By clicking a row containing a session, the lecturer can see the statistics of attendance of the session.

Of course in the backend, we have a stable and safe Firebase Database for our attendance taking app. This Firebase Database has the relevant information for all the classes, timetables and attendance records. The records are successfully updated, retrieved and maintained according to the requirements of our app. The Firebase Satabase also acts as a maintenance tool for the record keepers of the university in order to keep updated lecturer and student personnel files, their subjects for the semester and their attendance records of their years in the University. All data can be manipulated, including attendance records (for emergency sake) but the database can only be accessed by authorised personnel.

\section{Future Developments}
The application we have created is not perfect by far, therefore there will be new features and improvements added in the future. Integration to the Nottingham’s database is also required for this application to be functionally used as well as development for iOS devices. Due to time restrictions, there are also some features that we have considered adding, however, unable to, such as implementation of AndroidX and Junit testing. As of the future development of the application, many features have been implemented such as a cloud database messaging to start services, incorporating RSSI to measure distance between devices, allowing manual attendance taking and integrating the administrator’s website with the Nottingham’s administrator website. Finally, there is a feature that is being considered, which is the installation of an infrared device in each class to track student’s physical presence, however, this requires cooperation from the University of Nottingham for it to be successful.
\end{document}