\documentclass[../report.tex]{subfiles}
\begin{document}
\section{Using Android Devices and BLE}
There are some existing solutions that utilizes BLE to record attendance. One of them is to attach a sensor to each students’ identity card which could interact with the application installed in the lecturer’s Android device. This sensor contains a unique string that can be associated with the student card it is attached to. During class, the lecturer will open the application to scan the sensors to collect the students’ data into the application. To avoid attendance taken by proxy (students carrying more than one identity card or students standing outside the classroom during class), an infrared sensor could be installed in each classroom at the correct location and angle that can count the number of students who are physically present in the class. If the number of present students recorded by the application does not match the data recorded by the infrared sensor, the lecturer will receive an error message from the application \cite{7848166}.

Another way is to install BLE beacon in each classroom that transmit a ``magic number" to each nearby Android devices with the required application installed. Through a web-based attendance management system, the lecturer could set the ID and name of the class prior to the class. During class, the lecturer will turn on the BLE beacon to send a ``magic number" to students’ Android devices nearby. Furthermore, the lecturer also turns on registration for attendance on the web-based attendance management system, which offers some basic management features such as list of attendance rate and records and import, export, and manual alteration of attendance records. The application installed in students’ device allows the students to scan their student identity cards through Near Field Communication (NFC) reader. It is also not necessary to scan their cards with their own device and any device with the application installed could scan their student card. Meanwhile, the Android devices will receive the “magic number” from the BLE beacon. To record students’ attendance, the application will send the ``magic number" and the scanned student card and name to the server \cite{7350708}.

Other than that, Bluetooth beacons can be installed to each room so that students’ device can scan their presence and extract the beacon’s UUID. The students will install an application which logs them into their respective user account and scan their surroundings in search for Bluetooth beacon’s UUIDs. Each room will be represented by a beacon’s UUID. The administrator server will then match the extracted UUID, retrieval timestamp, and user account, all which are sent from the application installed in the student’s device, with the database to check if the student is present in the correct room during academic sessions \cite{20200029173}.

\section{Using BLE Indoor Positioning Technology}
There is also an alternative method of attendance management which does not require any mobile application. The university could install four Bluetooth station modules in strategic locations in a room. The Bluetooth module is a sensor system programmed in Python over Raspberry Pi which is also equipped with Bluetooth USB dongles. They are installed in the walls of each room to connect with powerlines and the Local Area Network (LAN). In this system, the lecturer could set the classes and list of students on the web-based attendance management system. On the other hand, students will also register the media access control (MAC) address of their device on the web-based system.

During academic sessions, the web-based system will send the list of student MAC address based on the timetable registered by the lecturer to the Bluetooth modules. The Bluetooth modules will then scan for Bluetooth devices nearby and detect their RSSI and MAC address. The RSSI data will be processed using fingerprint localization method based on Artificial Neural Network (ANN) to estimate the location of the student in the classroom and the MAC address can be associated with a student. The data collected will be sent to the server and matched with the records in the database to register the students’ attendance \cite{info11060329}.

\section{Android Bluetooth Low Energy}
Android introduced Bluetooth Low Energy (BLE) functionality in Android 4.3 (API level 18). BLE has a lower power consumption in comparison to Classic Bluetooth. The transfer of data (known as attributes) between two BLE capable devices is based around the Generic Attribute Profile (GATT) which is built on top of the Attribute Protocol (ATT) \cite{android-ble}. 

ATT defines a standard protocol for attribute transfer between BLE devices by defining how attributes are formatted for transfer. Each attribute is uniquely identified by a Universally Unique Identifier (UUID) which is a standardised 128-bit format for a string ID. Attributes that are transferred are formatted as either services, characteristics, or descriptors. A service is a collection of characteristics. A characteristic contains a single value and any number of descriptors which are attributes for describing the characteristic. These descriptors may specify the characteristic’s use, minimum and maximum values, unit of measurement, etc.
A GATT profile specifies what kind of attributes are transferred. Bluetooth SIG provides existing profiles (such as Alert Notification Profile and Heart Rate Profile) for common BLE devices, but custom GATT profiles may also be written by developers. Custom GATT profiles defines a new service and characteristics of the service.

Android documentation specifies roles taken by two BLE capable devices that are connected. Regarding making a connection between two BLE devices, there exists two roles, the central and peripheral roles. The device that acts as the central scans for devices by looking for advertisements of services. Once a device is found, the central device is responsible for initiating a connection. The peripheral device is responsible for sending out advertisements of its services defined by a GATT profile. After the connection is created, the devices take on two new roles, either GATT client or GATT server. Usually, the central device takes the client role while the peripheral takes the server role. The GATT client is responsible for making requests to read or write into the characteristics inside the server’s service which are defined by a GATT profile. The GATT server is responsible for notifying the client of characteristic changes and responding to client requests.

\section{Firebase}
Firebase is a platform developed by Google for creating and maintaining mobile and web applications as their Flagship product for app development \cite{firebase-crunchbase}. There are 18 products available split into 3 groups, namely Develop, Quality and Grow. For our application we are using Firebase to host our application as well as the Authentication and Realtime Database features available on the firebase platform. 

We decided to use Firebase Hosting to host our app not only because it’s free, but also because of the simplicity of using Firebase to understand our backend statistics such as the amount of data downloaded, the amount of storage used as well as the ease of using the built-in authentication features available. Since Firebase is cloud-hosted and free, it is both cost effective and the risk of losing data is minimal.

Firebase Authentication provides a way for developers to identify the users who access the application or website. There are many different types of authentication but project we have used the ``email and password" based authentication type. We have decided to make use of a function that allows users to create accounts either on the firebase console itself or in their mobile apps. Developers or System Administrators update or register data on the Firebase Realtime Database after successfully signing into the website. Every user account in the system has a corresponding User ID whether they are a lecturer or a student. Admin accounts also have User IDs but they are not useful in this project.

Firebase Realtime Database is a NoSQL cloud-hosted database whereby data as JavaScript Object Notation (JSON) tree and can be synchronised in realtime to every connected client. The database is structured as a JSON tree with parent and child nodes with their respective keys. For cross-platforms apps (e.g. Android and Web), every client shares the same data instance and simultaneously receive the same updates made in the database \cite{firebase-realtime-database}. We keep the data denormalized in the database so that the data can be downloaded efficiently as separate queries instead of fetching all the data nested in a particular location. This is because when client fetches data, all the child nodes are retrieved. If too much data is nested in a location, the client might end up downloading data which is not needed. Granting users read and write access to a location also grants read and write access to all child nodes, so we try to keep the data structure as flat as possible \cite{firebase-database-structure}.

Note that all the Firebase Services used within this project encrypt data in transit with HTTPS and in the Google Servers \cite{firebase-database-security}.

\section{Vue.js}
Vue is a framework for building user interface of a website and is especially useful in creating single-page application (SPA). Developers can write templates, which are based in HTML syntax, and bind them with the components instance data of a website. The component templates most useful in this project is a type called x-template, which allows the developer to write HTML code inside a script tag \cite{vue-js}. Instead of creating multiple html files and redirects, they can be written in a single JavaScript file or a single HTML file under script tags. They can also create a navigation bar and allow users to switch between different templates by using the Vue-router-library. Routes can be created as paths to templates so that the router can link to the items in the navigation bar to their respective templates \cite{vue-js-routing}. The main purpose of using Vue in this project is to allow the user to access multiple templates without refreshing the webpage.

\section{Selenium Browser Automation}
Selenium WebDriver simulates how a real user would use a browser, on local or remote machines. It works with all major browsers, including Chrome, Edge, Firefox, etc. This WebDriver also refers both to the language binding and implementations of the browser controlling code. It is an objected-oriented API designed as a compact programming interface \cite{selenium-docs}. In this project, we will use Python to implement the WebDriver and create a script to drive browser automation. The script will be used to populate the database and test the functionalities of the administration website.
\end{document}